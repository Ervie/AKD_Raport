\section{Użyte algorytmy}

\subsection{PNG}

\textbf{PNG} (ang. \textit{Portable Network Graphics}) -- to rastrowy format plików graficznych oraz system bezstratnej kompresji obrazów barwnych oraz w odcieniach szarości. Został opracowany w 1995 roku jako następca formatu GIF.
Cechy algorytmu PNG:

\begin{itemize}
	\item Kompresor PNG jest schematem predykcyjnym.
	\item Kompresja polega na wykonaniu transformacji każdej linii obrazu, po czym skompresowaniu go za pomocą algorytmu deflate.
	\item Transformacja polega na obliczeniu różnicy pomiędzy wartością piksela, a wartością obliczoną na podstawie funkcji przewidującej.
	\item Funkcje przewidujące to funkcje wyliczające wartości na podstawie wartości sąsiadujących pikseli.
	\item Stosowane jest kodowanie LZ77, a do kodowania położenia i długości dopasowanej długości używa sie kodów przedrostkoych zmiennej długości (predefiniowanych lub kodów Huffmana z ograniczeniem).
\end{itemize}

\subsection{JPEG-LS}

\textbf{JPEG-LS} jest nowym standardem komitetu JPEG dla kompresji bezstratnej obrazów wynalezionym w 1999 roku. Jest następcą Lossless JPG opartym o algorytm LOCO-I. Wykorzystywany jest ze względu na osiągane współczynniki kompresji (bliskie najlepszym) oraz dużą elastyczność umożliwiającą implementację rozszerzeń. Oprócz tego istnieje ,,prawie bezstratny'' (ang. \textit{nearly lossless}) wariant tego algorytmu.

Uogólniony schemat algorytmu JPEG-LS:

\begin{enumerate}
	\item \textbf{Wykrywanie krawędzi pionowych i poziomych} -- stosowany w tym celu jest prosty predykat nieliniowy (wyznaczanie jedynie przy pomocy stałopozycyjnych operacji dodawania/odejmowania i porównań).
	\item \textbf{Kodowanie} -- przy pomocy kodera entropijnego i zmodyfikowanej (uproszczonej) rodziny kodów Rice'a. Na tym etapie koduje się również błędy predykcji.
	\item \textbf{Modelowanie kontekstu} -- wyznaczony na podstawie 3 gradientówModel jest dedykowany dla kodów rodziny Rice'a.
\end{enumerate}

\subsection{JPEG2000}

\textbf{JPEG-2000} podobnie jak JPEG-LS jest następcą algorytmu JPEG wyłonionym w drodze konkursu. Opracowany w 2000 roku, stanowi standard komitetu JPEG zarówno dla stratnej, jak i bezstratnej kompresji obrazów. Jest algorytmem przeznaczonym dla obrazów barwnych i w odcieniach szarości opartym o transformatę falkową. Pozwala na kodowanie piramidowe i progresywne.

Uogólniony schemat algorytmu JPEG2000:

\begin{enumerate}
	\item \textbf{Sprowadzenie nominalnego zakresu jasności pikseli do przedziały symetrycznego względem 0} -- jest to krok tożsamy z tym wykonywanym w algorytmie JPEG.
	\item \textbf{Transformacja przestrzeni barw} -- może być nieodwracalna lub odwracalna. Ta pierwsza możliwa jest tylko razem z nieodwracalną transformatą falkową.
	\item \textbf{Podział składowej na kafelki} -- realizowany poprzez nałożenie siatki prostokątnej na obraz (najczęściej o wymiarach 256 na 256 pikseli). Krawędzie siatki nie muszą się pokrywać z krawędziami obrazu, a obraz może mieścić się w jednym kafelku.
	\item \textbf{Wyznaczenie transformaty falkowej dla każdego kafelka} -- przed transformatą należy ekstrapolować wiersze. W przypadku kompresji bezstratnej, stosujemy transformatę odwracalną.
	\item \textbf{Kwantyzacja współczynników transformaty} -- dla kompresji bezstratnej krok zawsze wynosi 1. W przypadku operacji kompresji stratnej koder dobiera krok, aby możliwe było osiągnięcie zadanego współczynnika.
	\item \textbf{Kodowanie skwantowanych współczynników} -- przy pomocy kodera arytmetycznego.
	\item \textbf{Zapisanie zakodowanych danych w pliku o strukturze opisej przez standard}
\end{enumerate}