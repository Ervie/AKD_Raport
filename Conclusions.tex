\section{Wnioski}

Wszystkie z testowanych formatów generowane są w trakcie kompresji bezstratnej. Jest to bezpośredni powód, dla którego uzyskane współczynniki kompresji są gorsze niż w przypadku kompresji stratnej. Istnieją jednak zalety korzystania z kompresji bezstratnej -- ponieważ nie tracimy danych, są to procesy w pełni odwracalne. W subiektywnym odczuciu różnice między skompresowanymi obrazami nie są widoczne gołym okiem.

Przy porównaniu wyników uzyskanych w trakcie kompresji obrazów barwnych, rzuca się w oczy przede wszystkim różnica w wartości współczynnika dla obrazów \textit{clegg}, \textit{frymire} oraz \textit{serrano}. Powodem tej dysproporcji jest pochodzenie obrazów źródłowych -- wszystkie 3 są obrazami stworzonymi cyfrowo. W takim przypadku górę bierze format PNG ze względu na swój algorytm predykcji. W przypadku rysowanych obrazów, częstym scenariuszem jest występowanie w sąsiedztwie pikseli o identycznej wartości, co pozwala PNG na łatwiejsze enkodowanie dużych porcji danych. W takich przypadkach osiągał 3-4 razy lepsze wyniki niż JPEG2000 i 1,5-2 razy lepsze wyniki niż JPEG-LS. W pozostałych przypadkach (\textit{lena3}, \textit{monarch}, \textit{peppers3}, \textit{sail} oraz \textit{tulips}) mamy do czynienia ze zdjęciami -- w takim scenariuszach wyniki są bardziej wyrównane, jednak najlepsze współczynniki uzyskiwał algorytm JPEG2000, a najgorzej wypadał PNG. Jedynym wyjątkiem jest obraz \textit{lena3}, dla którego JPEG-LS uzyskał minimalnie lepszy wynik. Przeciętnie najlepiej wypadł format PNG, jednak zawdzięcza to przede wszytkim kompresji obrazów rysowanych.

Nieco inaczej wygląda sytuacja w przypadku obrazów w odcieniach szarości. Znaczące różnice (ponownie na korzyść PNG) można dostrzec głównie dla obrazów \textit{france} i \textit{washsat}. Pierwszy z nich jest slajdem prezentacji (utworzonym komputerowo), a nieznaczne zmiany w kolorystyce kolejnych pikseli ponownie faworyzuje algorytm PNG. Ciekawszy jest drugi przypadek, przedstawiający krajobraz z lotu ptaka -- jest to zdjęcie, jednak charakteryzujące się niewielkim kontrastem pomiędzy pikselami. W pozostałych przypadkach uzyskiwane wyniki były bardzo zbliżone, przeciętnie najlepszy okazywał się format JPEG-LS.

Różnice w szybkości są odczuwalne (zwłaszcza dla większych obrazów kolorowych) -- o ile JPEG-LS był kilka razy szybszy niż PNG, znaczący przeskok zanotowano pomiędzy tymi 2 algorytmami a JPEG2000 - ten ostatni był nawet 10 razy wolniejszy niż JPEG-LS. Dla obrazach w odcieniach szarości różnice te były mniejsze (co jednak jest też spowodowane mniejszym rozmiarem grafik), jednak proporcje pozostały te same -- najszybciej wykonywała się kompresja do formatu JPEG-LS, wyprzedzając PNG i JPEG2000.