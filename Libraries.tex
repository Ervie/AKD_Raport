\section{Wykorzystane biblioteki}

\subsection{PNG -- pnmtopng}

\subsubsection{Instalacja}

\textbf{pnmtopng} jest elementem pakietu NetPBM. Aby umożliwić korzystanie z biblioteki na systemie Unixowym, należy:

\begin{enumerate}
	\item Posiadać zainstalowane wymagane LibPNG, ZLIB, dowolny kompilator języka C oraz Perl w wersji 6.0 lub nowszy.
	\item Pobrać pliki źródłowe spakowane do formatu .tar ze strony SourceForge (\url{https://sourceforge.net/projects/netpbm/files/}).
	
	\item Wypakować pliki do wybranego przez siebie folderu.
	
	\item Wykonać komendy
	\begin{lstlisting}[language=bash]
	./configure
	make package
	./installnetpbm
	\end{lstlisting}
\end{enumerate}

W przypadku Windowsa, należy posłużyć środowiskami Cygwin lub Djgpp.

Prostszą alternatywą jest pobranie skompilowanych plików binarnych ze strony: \url{http://gnuwin32.sourceforge.net/packages/netpbm.htm}

\subsubsection{Uruchamianie}

Aby uruchomić kompresję plików, należy posłużyć się aplikacją \textbf{pnmtopng}. Przykładowe wywołanie programu w celu kompresji pliku \textit{clegg.pp\textsl{}m}:

\begin{lstlisting}[language=bash]
	pnmtopng clegg.pnm >clegg.png
\end{lstlisting}

\subsection{JPEG-LS -- SPMG/JPEG-LS}

\subsubsection{Instalacja}

\subsubsection{Uruchamianie}

\subsection{JPEG2000 - JasPer}

\subsubsection{Instalacja}

\subsubsection{Uruchamianie}