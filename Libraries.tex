\section{Wykorzystane biblioteki}

\subsection{PNG -- pnmtopng}

\subsubsection{Instalacja}

\textbf{pnmtopng} jest elementem pakietu NetPBM. Aby umożliwić korzystanie z biblioteki na systemie Unixowym, należy:

\begin{enumerate}
	\item Posiadać zainstalowane wymagane LibPNG, ZLIB, dowolny kompilator języka C oraz Perl w wersji 6.0 lub nowszy.
	\item Pobrać pliki źródłowe spakowane do formatu .tar ze strony SourceForge (\url{https://sourceforge.net/projects/netpbm/files/}).
	
	\item Wypakować pliki do wybranego przez siebie folderu.
	
	\item Wykonać komendy
	\begin{lstlisting}[language=bash]
	./configure
	make package
	./installnetpbm
	\end{lstlisting}
\end{enumerate}

W przypadku Windowsa, należy posłużyć środowiskami Cygwin lub Djgpp.

Prostszą alternatywą jest pobranie skompilowanych plików binarnych ze strony: \url{http://gnuwin32.sourceforge.net/packages/netpbm.htm}

\subsubsection{Uruchamianie}

Aby uruchomić kompresję plików, należy posłużyć się aplikacją \textbf{pnmtopng}. Przykładowe wywołanie programu w celu kompresji pliku \textit{clegg.pp\textsl{}m}:

\begin{lstlisting}[language=bash]
	pnmtopng clegg.pnm >clegg.png
\end{lstlisting}

\subsection{JPEG-LS -- SPMG/JPEG-LS}

\subsubsection{Instalacja}

\subsubsection{Uruchamianie}

\subsection{JPEG2000 -- JasPer}

\subsubsection{Instalacja}

\textbf{JasPer} jest otwarto źródłową biblioteką zawierającą implementację algorytmu JPEG2000. W celu instalacji na systemie Windows należy:

\begin{enumerate}
	\item Pobrać i wypakować pliki źródłowe ze strony projektu (\url{http://www.ece.uvic.ca/~frodo/jasper/}) do wybranej przez siebie lokalizacji.
	
	\item Utworzenie dodatkowy zmiennych środowiskowych:
	\begin{enumerate}
		\item \textbf{\%SOURCE\textunderscore DIR\%} -- katalog nadrzędny, w którym wypakowane zostały pobrane pliki
		\item \textbf{\%BUILD\textunderscore DIR\%} -- ścieżka do katalogu używanego do zbudowania aplikacji
		\item \textbf{\%INSTALL\textunderscore DIR\%} -- ścieżka do katalogu używanego do zainstalowania aplikacji
	\end{enumerate}
	Zmienne te są zdefiniowane pliku make i będą wykorzystywane w trakcie instalacji.
	
	\item W wierszy poleceń wykonać polecenie:
	
	\begin{lstlisting}[language=bash]
	cmake -help
	\end{lstlisting}
	
	Pozwala to podejrzeć nazwy wszystkich dostępnych generatorów (programów umożliwiających kompilację plików źródłowych).
	
	\item Za pomocą wybranego przez siebie generatora (parametr -G) wykonać komendę tworzącą plik solucji .sln.
	
	\begin{lstlisting}[language=bash]
	cmake -G "Visual Studio 14 2015 Win64" -H%SOURCE_DIR% -B%BUILD_DIR% ^ -DCMAKE_INSTALL_PREFIX=%INSTALL_DIR%
	\end{lstlisting}
	
	\item W wierszu poleceń programisty (ang. \textit{Developer Command Line}) wykonać komendę:
	
	\begin{lstlisting}[language=bash]
	msbuild %build_dir%\INSTALL.vcxproj
	\end{lstlisting}
	
	Spowoduje to utworzenie skompilowanego programu jasper.exe w ścieżce podanej w \%INSTALL\textunderscore DIR\%.
\end{enumerate}



\subsubsection{Uruchamianie}

Aby uruchomić program jasper.exe, należy w pierwszej kolejności skopiować do tego samego katalogu bibliotekę \textit{libjasper.dll}, wygenerowaną wcześniej w folderze \textit{lib}. Może również dojść do sytuacji, w jakiej nie zostanie wykryta biblioteka \textit{ucrtbased.dll}. W takim przypadku należy pobrać ją z internetu i wypakować ją w \textit{$C:\backslash Windows\backslash System32\backslash$}. Przykładowa komenda uruchamiająca kompresję (plik clegg.ppm):

\begin{lstlisting}[language=bash]
jasper.exe --input clegg.ppm --output clegg.jp2
\end{lstlisting}